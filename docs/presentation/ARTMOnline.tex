\documentclass{beamer}
\usepackage[english,russian]{babel}
\usepackage{amsmath,mathrsfs,mathtext}
\usepackage{graphicx, epsfig}
\usepackage[utf8]{inputenc} 
%\usepackage{amssymb,amsfonts,amsmath,mathrsfs,mathtext}
\usepackage{epstopdf}
\usepackage{subfig}
\usepackage[labelformat=empty]{caption}
\usepackage{wasysym} 

 


 
\beamertemplatenavigationsymbolsempty
\usetheme{Madrid}%{Warsaw}%{Singapore}%{Darmstadt}
\usecolortheme{sidebartab}
%\definecolor{beamer@blendedblue}{RGB}{15,120,80}
%----------------------------------------------------------------------------------------------------------
\title{ARTM Online}
\subtitle{Технология интерактивной визуализации тематических моделей}

\institute[МФТИ]{Московский физико-технический институт}
\author{Дмитрий Федоряка}

%\date{\footnotesize{\emph{Курс:} Машинное Обучение и Анализ Данных\par Группа 374, 2016 Весна}}


%----------------------------------------------------------------------------------------------------------
\begin{document}
%----------------------------------------------------------------------------------------------------------
\begin{frame}
%\thispagestyle{empty}
\titlepage
\end{frame}
%-----------------------------------------------------------------------------------------------------
\begin{frame}{Цели}
		\begin{itemize}  
		\item Инструмент для исследователей BigARTM
		\item Сайт для рекламы наших возможностей
		\item Среда для сбора оценкок ассесоров
		\end{itemize}
\end{frame} 

\begin{frame}{Визуализации}
	\begin{itemize}  
			\item Документ
			\item Тема
			\item Термин
			\item Иерархия
			\item Темпоральная визуализация
	\end{itemize}
\end{frame} 

\begin{frame}{Требования}
	\begin{itemize}  
			\item Универсальность и простота.
			\item Простое добавление модулей (визуализация, именование тем).
			\item Визуальная привлекательность (d3).
			\item Скорость и масштабируемость (БД, AJAX).
	\end{itemize}
\end{frame} 


\begin{frame}{Последовательность построения модели}
		\begin{itemize}  
		\item Коллекция $\to$ мешок слов
		\item Мешок слов $\to$ батчи ARTM
		\item батчи ARTM $\to$ Построение модели $\to$ $\Phi, \Theta, \Psi$
		\item $\Phi, \Theta, \Psi$ $\to$ дерево тем и документов.	
		\item Автоматическое именование тем.
		\item Визуализация.
		\end{itemize}
\end{frame} 

\begin{frame}{Решение}
		\begin{itemize}  
			\item Веб-сервис на python и django с базой данных.
			\item Все объекты (документы, темы, термины, отношения) храним в базе данных
			\item Текст документов -- в файлах.
			\item Есть интерфейс построения моделей, но можно их строить отдельно и загружать.
			\item Визуализации - как отдельные модули (2 файла: py, js).
		\end{itemize}
\end{frame} 

\begin{frame}{Показать сайт}
	
\end{frame} 


\begin{frame}{Планы}
		\begin{itemize}  
			\item Упорядочивание тем (в процессе)
			\item Именование тем
			\item Интерфейс ассесора (?)
		\end{itemize}
\end{frame} 

\end{document} 